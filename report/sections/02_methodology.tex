\section{Methodology}

To translate raw administrative data into actionable governance insights, we developed two proprietary metrics: the \textbf{Migration Intensity Index (MII)} and the \textbf{Bio-Update Lag (BUL)}. These metrics allow us to normalize district-level data and identify anomalies.

\subsection{Migration Intensity Index (MII)}
This metric identifies districts acting as \enquote{magnets} for migrant workers. It hypothesizes that a district with low natural growth (low child enrolment) but high adult activity (high demographic updates) is a migration destination.

\begin{equation}
    MII_{d} = \frac{\text{Adult Demo Updates}_{d}}{\text{Child Enrolment}_{d} + 1}
\end{equation}

Where:
\begin{itemize}
    \item $MII_{d}$ is the index for district $d$.
    \item A high score ($> 5.0$) indicates a \textbf{Migrant Destination Hub}.
    \item A low score ($< 0.5$) indicates a \textbf{Rural Region}.
\end{itemize}

\subsection{Bio-Update Lag (BUL)}
This metric identifies compliance gaps. According to UIDAI regulations, children must update biometrics at age 5 and 15. We compare the mandatory update volume against the district's baseline enrolment history.

\begin{equation}
    BUL_{d} = \frac{\text{Biometric Updates (5-17)}_{d}}{\text{Child Enrolment (0-5)}_{d}}
\end{equation}

\noindent A low $BUL$ score signals that a district has a large population of children who are failing to complete their mandatory updates, putting them at risk of service disruption.

\subsection{Technical Implementation}
We implemented this logic using Python (Pandas). The data was aggregated by district to smooth out daily volatility.

\noindent
\begin{minipage}{\linewidth}
\begin{lstlisting}[language=Python, caption=Calculating the Migration Index]
# Feature Engineering: Migration Intensity Index
# We add +1 to the denominator to handle edge cases with 0 enrolments

df['migration_index'] = df['demo_update_adult'] / (df['enrol_child'] + 1)

# Identifying the top 'Magnet' districts
migrant_hubs = df.sort_values(by='migration_index', ascending=False)
\end{lstlisting}
\end{minipage}

\subsection{Outlier Detection}
We applied a ranking algorithm to filter the top 20 and bottom 20 districts for both metrics. This isolates the \enquote{extreme cases} that require immediate administrative intervention, filtering out average-performing districts to focus resources where they are needed most.